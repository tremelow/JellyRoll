\markboth{}{}
% Plus petite marge du bas pour la quatrième de couverture
% Shorter bottom margin for the back cover
\newgeometry{inner=30mm,outer=20mm,top=40mm,bottom=20mm}

%insertion de l'image de fond du dos (resume)
%background image for resume (back)
\backcoverheader

% Switch font style to back cover style
\selectfontbackcover{ % Font style change is limited to this page using braces, just in case

\titleFR{Méthodes d'analyse asymptotique et d'approximation numérique -- Problèmes d'évolution multi-échelles de type oscillatoire ou dissipatif}

\keywordsFR{décomposition micro-macro, précision uniforme, relaxation rapide, moyennisation}

\abstractFR{Les problèmes à relaxation rapide apparaissent dans de nombreux systèmes physiques ou biologiques, notamment dans le cadre de modèles cinétiques avec collisions. Leur comportement mélange une dynamique de relaxation de temps caractéristique $\eps$ et une partie lente d'interactions (généralement non-linéaire) ou de transport. Naturellement, on cherche à résoudre ce type de problème numériquement. Malgré le développement depuis les années 1980 de méthodes de résolution adaptées peu coûteuses (i.e. stables et essentiellement explicites), un problème demeure: la précision des méthodes est dégradée lorsque le pas de discrétisation est d'ordre~$\eps$. Dans ce manuscrit, on présente une méthode pour dépasser cette limite. L'approche mise en \oe{}uvre consiste à effectuer des développements asymptotiques par rapport au paramètre~$\eps$ de sorte à pouvoir séparer le modèle asymptotique et son erreur; on parle alors d'un problème micro-macro. Ce nouveau problème peut être résolu avec une précision indépendante du paramètre~$\eps$. Nos développements asymptotiques font appel à des résultats récents de moyennisation, si bien qu'un chapitre de ce manuscrit est dédié à l'exposition de preuves originales de certains résultats de moyennisation connus. On discute en outre d'extensions possibles de nos résultats. }



\titleEN{Some methods of asymptotic analysis and numerical approximation -- Multi-scale evolution problems, of oscillary or relaxation type}

\keywordsEN{de 3 \`{a} 6 mots clefs}

\abstractEN{Stiff relaxation problems appear in numerous physical and biological systems, most notably in kinetic models with collisions. In the solution of such problems, two dynamics are intertwined: a relaxation of characteristic time~$\eps$, and a slow part of interactions or transport. We are interested in solving these problems numerically. Since the 1980s, efficient and stable methods have been developed, however one issue remains: the accuracy of the method degrades when the time-step is of size~$\eps$. In this work, we present a method to overcome this limit. Our approach consists in performing asymptotic developments with relation to~$\eps$ in order to consider an asymptotic model and its error separately -- we call this a micro-macro decomposition. This new problem may be solved with an accuracy independent of~$\eps$. As our asymptotic developments use results from averaging, a chapter of this work is dedicated to averaging results. Specifically, we present original proofs of known results, using recent frameworks which make algebraic reasonings straightforward. A brief discussion surrounding possible extensions of our results is conducted at the end of the manuscript.}

}

% Rétablit les marges d'origines
% Restore original margin settings
\restoregeometry
