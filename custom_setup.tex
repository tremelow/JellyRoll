\usepackage{amsmath,amsfonts,amssymb,amsthm}
\usepackage{empheq}
\usepackage{subcaption}
\usepackage{enumerate}  % lists with (i)
\usepackage{yhmath}     % for very wide hats
\usepackage{ifthen}
\usepackage{tikz-cd}

% \usepackage[
%   backend=biber,
%   style=alphabetic, % `numeric` to count 
%   sorting=nyt, 
%   maxalphanames=99, 
%   maxbibnames=99,
%   doi=false,isbn=false,url=false
% ]{biblatex}
\addbibresource{bibJellyRoll.bib}

\NewBibliographyString{toappear}
\DefineBibliographyStrings{english}{%
  toappear = {to appear in},
}
\DefineBibliographyStrings{french}{%
  toappear = {à paraître dans},
}

\renewbibmacro*{in:}{%
  \iffieldundef{pubstate}
    {}
    {\printfield{pubstate}%
     \setunit{\addspace}%
     \clearfield{pubstate}}%
%   \printtext{%
%     \intitlepunct}%
    }


%----------------------------------------------------------
% Temporary packages
\usepackage{lipsum}
\usepackage[textwidth=3.7cm]{todonotes}
% \geometry{inner=1cm, outer=4.5cm}
% \setlength{\marginparwidth}{3.7cm}
%----------------------------------------------------------

\newtheorem{theorem}{Theorem}[section]
\newtheorem{proposition}[theorem]{Proposition}
\newtheorem{definition}[theorem]{Definition}
\newtheorem{remark}[theorem]{Remark}
\newtheorem{corollary}[theorem]{Corollary}
\newtheorem{lemma}[theorem]{Lemma}
\newtheorem{assumption}[theorem]{Assumption}
\newtheorem{property}[theorem]{Property}

\newtheorem*{theorem*}{Theorem}
\newtheorem*{FRtheorem*}{Théorème}
\newtheorem*{proposition*}{Proposition}
\newtheorem*{definition*}{Definition}
\newtheorem*{FRdefinition*}{Définition}
\newtheorem*{remark*}{Remark}
\newtheorem*{FRremark*}{Remarque}
\newtheorem*{corollary*}{Corollary}
\newtheorem*{lemma*}{Lemma}
\newtheorem*{assumption*}{Assumption}
\newtheorem*{FRassumption*}{Hypothèse}
\newtheorem*{property*}{Property}
\newtheorem*{FRproperty*}{Propriété}



\newcommand{\eps}{\ensuremath{\varepsilon}}
\newcommand{\pa}{\ensuremath{\partial}}
\newcommand{\dd}{\ensuremath{\mathrm{d}}}
\newcommand{\rk}[1]{^{[#1]}}
\newcommand{\bigO}{\mathcal{O}}
\newcommand{\Dt}{\ensuremath{\Delta t}}

\newcommand{\setN}{\mathbb{N}}
\newcommand{\setZ}{\mathbb{Z}} 
\newcommand{\setQ}{\mathbb{Q}}
\newcommand{\setR}{\mathbb{R}} 
\newcommand{\setC}{\mathbb{C}}
\newcommand{\setT}{\mathbb{T}}
\newcommand{\setK}{\mathcal{K}}

\newcommand{\R}{\mathbb{R}} \newcommand{\C}{\mathbb{C}}
\newcommand{\Z}{\mathbb{Z}} \newcommand{\N}{\mathbb{N}}
\newcommand{\T}{\mathbb{T}}

\newcommand{\RR}{\rangle}
\newcommand{\LL}{\langle}
\newcommand{\bRR}{\big \rangle}
\newcommand{\bLL}{\big \langle}
\newcommand{\BRR}{\Big \rangle}
\newcommand{\BLL}{\Big \langle}

\newcommand{\among}[2]{\begin{pmatrix} #1 \\ #2 \end{pmatrix}}

% triple norm 
\newcommand{\vertiii}[1]{{\left\vert\kern-0.25ex\left\vert\kern-0.25ex\left\vert #1 %
    \right\vert\kern-0.25ex\right\vert\kern-0.25ex\right\vert}}

\DeclareMathOperator{\id}{id}
\DeclareMathOperator{\tr}{tr}
\DeclareMathOperator{\Tr}{tr}
\DeclareMathOperator{\err}{err}

\makeatletter
\newcommand*\bigcdot{\mathpalette\bigcdot@{.5}}
\newcommand*\bigcdot@[2]{\mathbin{\vcenter{\hbox{\scalebox{#2}{$\m@th#1\bullet$}}}}}
\makeatother

\def\checkmark{\tikz\fill[scale=0.4](0,.35) -- (.25,0) -- (1,.7) -- (.25,.15) -- cycle;} 

%--------------------------------
% Pour Chap. III
\newcommand{\K}{\mathcal{K}}
\newcommand{\ul}[1]{\underline{#1}}
\newcommand{\ol}[1]{\overline{#1}}

\newcommand{\e}{\varepsilon}
\newcommand{\ee}{^{\e}}
\newcommand{\ie}{_{\e}}
\newcommand{\eeps}{\ee}
\newcommand{\ieps}{\ie}
\newcommand{\inveps}{\frac{1}{\eps}}

\newcommand{\lp}{\big(}\newcommand{\rp}{\big)}
\newcommand{\LP}{\Big(}\newcommand{\RP}{\Big)}
\newcommand{\Langle}{\big\langle}
\newcommand{\Rangle}{\big\rangle}
\newcommand{\D}{\mathrm{d}}

\newcommand{\dpt}{\partial_t}
\newcommand{\dptau}{\partial_{\tau}}
\newcommand{\dptheta}{\partial_{\theta}}
\newcommand{\dpu}{\partial_{u}}
\newcommand{\dpx}{\partial_{x}}
\newcommand{\dpz}{\partial_{z}}
\newcommand{\dpy}{\partial_{y}}