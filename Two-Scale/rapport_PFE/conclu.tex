\chapterb{Conclusion}

On a présenté le modèle et les problèmes qu'il engendrait d'un point de vue numérique. 
On a ensuite effectué un développement double-échelle pour séparer la dynamique rapide $e^{-t/\epsilon}$ de la dynamique lente. 
Après cette séparation, on a exhibé des conditions nécessaires à la régularité du nouveau problème, et on a montré que sous ces conditions, le problème était bien posé. 
Une fois cela fait, on a trouvé des conditions initiales et au bord qui permettent d'assurer que $X,Z$ et leurs dérivées première et seconde en $t$ sont uniformément bornées. 
À partir de ces données, on a construit un schéma numérique uniformément précis qu'on a implémenté et dont on a évalué les performances et les limites en le comparant à d'autres approches moins coûteuses mais moins (voire pas) robustes. 
Les résultats étant insatisfaisant en termes de performance et d'exportabilité, on a rapidement présenté deux de nos pistes de recherche qui seront poursuivies en thèse. \\


Il est très facile de se perdre dans les différentes grandeurs en jeu dans ce problème: on a affaire à des séries en puissance de $\epsilon$ dont les coefficients sont des séries exponentielles en $\tau$, et on doit faire varier tout cela en fonction de $t$. 
On peut aussi facilement confondre régularité et caractère uniformément borné des dérivées, ce qui pose problème étant donné l'importance des conditions de compatibilité. 

Comme on l'a évoqué rapidement, il peut être intéressant de travailler avec des séries exponentielles, mais il n'est pas évident de construire une implémentation se basant uniquement là-dessus, et il faut se rendre compte que les coefficients en jeu sont plutôt instables ---d'où notre normalisation en section \ref{subsec:series}. 

Tous ces <<~pièges~>> sont autant de raisons qui ont rendu le début de ce stage un peu laborieux. 
Au bout d'un moment, la formalisation s'est mise en place et donc on a pu progresser relativement rapidement vers la fin. Les notions restent cependant complexes et nécessitent de prendre encore un peu de recul. \\


Le mélange d'algèbre, analyse, calcul formel et implémentations numériques de ce sujet est ce qui le rend tout à fait attirant pour une poursuite en thèse. 
Il est en outre apparent qu'il y a encore beaucoup de pistes à creuser en matière d'approche et d'implémentation et que donc ce sujet <<~préliminaire~>> à l'étude des EDP est déjà très intéressant en soi et motive à l'approfondir. 