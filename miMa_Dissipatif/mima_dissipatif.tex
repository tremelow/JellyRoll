
\clearemptydoublepage
\chapter{Convergence uniforme pour un problème dissipatif}

Ce chapitre reprend un article à paraître dans \textit{Mathematics of
Computation}, intitulé 
\begin{center}\itshape%
  A uniformly accurate numerical method for a class of dissipative
  systems
\end{center}
Dans cet article, on ...



\section{Construction et résultat de précision uniforme}

\subsection{Hypothèses et définitions}

Hyp : A diagonale avec des vap entières

Hyp : f analytique autour de (x,0)

Justification de l’hypothèse avec le théorème de variété centrale

Introduction des ensembles $\mathcal{K}_{\rho}$

Def : $R$ + borne sur $f$ et ses dérivées

Def : normes


\subsection{Construction d'un développement asymptotique}

Équation homologique

Définition de F et du projecteur

Relation de récurrence avec condition de fermeture

Def : défaut $\eta$

Thm : Bonne définition des morphismes et bornes


\subsection{Problème micro-macro et précision uniforme}

Obtention du problème micro-macro

Thm : problème micro-macro bien posé avec dérivées bornées

Rq : calcul de la donnée initiale

Présentation des schémas exponentiels

Def : norme associée aux schémas

Thm : précision uniforme :D

Rq : adaptation du résultat aux schémas IMEX-BDF


\section{Preuves des théorèmes}

\subsection{Bonne définition des morphismes, et leurs bornes}

Définition des morphismes périodiques

Filtrage des morphismes périodiques

Prop : hypothèses de UA périodique vérifiées

Thm de bonnes propriétés sur les morphismes périodiques

Principe du maximum pour conclure la preuve


\subsection{Caractère bien posé du problème micro-macro}

Caractère bien posé sur $v$ : $v(0)$ est gentil et $v(t)$ aussi par
Gronwall. $\Omega(v)$ est okay

Borner le terme linéaire $L$, puis Gronwall avec la source pour $w$ bien
posé


\subsection{Précision uniforme avec les schémas exponentiels}

Séparation entre $v$ et $w$

Partie $v$ bornée classique, passage à norme modifiée grâce à $\Omega$

Partie $w$ application directe des bornes de Hochbruck et Ostermann



\section{Extension partielle à des EDP discrétisées}

Présentation rapide du système

Disclaimer comme quoi on les regarde post-discrétisation


\subsection{Équation de télégraphe}

Transformation en Fourier

Tentative directe d’obtenir la variable en $-1/\varepsilon$ -> échec

Relaxation pour obtenir la variable $z$, montrer que ça va mieux

Développement à l’ordre zéro, montrer que ça se passe okay grâce à
$\exp{-tA/\varepsilon}$

Passage à l’ordre 1, observation comme quoi $-A/\varepsilon + F^{[1]}$
est mal posé

Relaxation dans le changement de variable, montrer que tout se passe
mieux

Eq : $\eta^{[1]}$

Prop : si les fréquences sont bornées, on peut faire du num tranquille
et l’erreur est indépendante de $\varepsilon$

Rq : les fréquences bornées c’est plus une CFL qu’un souci avec
$\varepsilon$


\subsection{Relaxation hyperbolique}

Présentation du système + condition de stabilité

Rq : relaxation indépendante de l’espace

Volumes finis avec Upwind, avec la variable $z$ naturelle

Développements ordre 1 parachutés avec relaxation (justif de télégraphe)

Rq : passage au continu

Commentaire sur le coût de résolution (indépendant de $\varepsilon$
malgré la relaxation)

Eq : défauts $\eta^{[0]}$ et $\eta^{[1]}$




\section{Résultats numériques}

\subsection{Applications directes}

Problème jouet lentement oscillant :

Exposition, dev à l’ordre 1 puis ordre 2

Rq : on peut retrouver la variété centrale

Rappel pb micro-macro, et commentaire sur le besoin de calcul explicite de $L$ pour l’arrondi

Problème inspiré de l’hyperbolique :

Exposition, passage aux variables $x, z$

Eq : dev à l’ordre 1, ordre 2 trop lourd

Choix de la fonction $g$ et de la donnée initiale

Résultats :

Figures : err. sup. sur $\varepsilon$ à gauche pour syst. original (ERK2), miMa 1 (ERK2) et miMa 2 (ERK3), et err. en fonction de $\varepsilon$ à droite (miMa 2 ERK3), pour différents $\Delta t$

=> Mise en avant et description du phénomène de réduction d’ordre

=> Observation de l’ordre de convergence grâce au problème micro-macro, ordres 1 et 2

\todo[inline]{Ajouter simus avec IMEX-BDF}


\subsection{EDP discrétisées}

Télégraphe : rappel de l’équation + donnée initiale

Résultats : réduction d’ordre, discussion sur le côté “erreur uniforme”

Relaxation hyperbolique : rappel de l’équation + donnée initiale

Résultats avec disclaimer qu’on regarde pas l’erreur en espace

Figures : même format que pour les EDOs mais on s’arrête à miMa 1
(forcément)


\subsection{Commentaires sur extensions directes}

\todo[inline]{Dans la discussion}