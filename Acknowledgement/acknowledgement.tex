\chapter*{Remerciements}


En premier lieu, c'est cliché mais je tiens à remercier mes parents. Malgré votre incompréhension de mon sujet de thèse, vous êtes restés curieux non seulement de l'avancée de mes recherches, mais aussi de certains concepts mathématiques sous-jacents que j'expliquais pourtant laborieusement. Merci pour votre soutien permanent, pour les bouffées d'air frais à la maison, pour les conseils de pédagogie, pour les coups de pied au fesses lorsque je laissais traîner des démarches, et surtout merci pour les valeurs que vous m'avez transmises. Merci aussi à Jules de me permettre d'échapper à mon monde trop scientifique, notamment à travers des aspects artistiques de la simulation. Une grosse pensée, bien sûr, à toute ma famille et tout particulièrement à Grand-Mère, qui aurait aimé voir cette thèse atteindre sa conclusion.

Vu les circonstances des dernières années, il semble naturel de remercier aussi en priorité mes camarades de premier confinement. À Léo pour les éclairages mathématiques, les balades à Regnéville et toutes les soirées ensemble. À Marie pour les dessins, les concerts (trop peu nombreux), et tout ce que tu m'apportes en tant que personne. Merci aussi aux autres rencontres rennaises de ces trois ans, notamment Louis, Antoine et Josselin, qui m'impressionneront toujours par leur compétence et leur humanité. Aux membres de REN, merci de m'avoir fait découvrir la pratique de l'escalade dans un cadre aussi bienveillant et chaleureux. On se retrouve sur les voies!

Merci à mes amis de Nantes, pour m'avoir supporté toutes ces années. Maël et Ece, qui continuez à me suivre même jusqu'à Grenoble et sur qui je peux toujours compter. Meven pour les sorties escalade et ta vision des mathématiques toujours aussi captivante. Paul, on ne discute jamais assez, mais tu es un modèle par ton investissement dans tes projets personnels et dans tes enseignements. Bon courage pour ta thèse, je crois en toi.

C'est une période particulière pour notre groupe d'amis de l'ENSTA~: certain$\cdot$e$\cdot$s démarrent à peine leur vie professionnelle pendant que d'autres développent déjà leur carrière. Je ne pourrais néanmoins pas espérer naviguer cette transition mieux entouré qu'en votre compagnie. Julien, exemplaire sur tous les aspects~; Aliénor, admirable par sa détermination (même si elle n'admettra jamais être \enquote{déterminée})~; Pauline pour ses super câlins, et pour avoir récemment fait le grand saut pour devenir un train (bravo)~; Lukô, une des personnes les plus adorables que je connaisse et qui mérite qu'on le lui rappelle~; Zoé, pour les photos de renards et le soutien constant~; Aurore, pour apporter du mignon dans un monde scientifique souvent aseptisé~; Servane, qui oublie trop souvent à quel point son investissement sur tous les fronts (politique, professionnel, même dans ses loisirs) est remarquable~; Adrien, qui, j'en suis sûr, va réussir à apporter sa manière de voir les maths à son sujet de thèse et y trouver satisfaction~; Léa, qui m'a accompagné pour le début de rédaction de ce manuscrit~; Benoît, Juliette, Rémi, Manu, Youssef, BPJ, Quentin, tant de gens qui participent à construire une communauté vivante et qui méritent mieux qu'une simple énumération en fin de liste (my bad).

Un grand merci aussi à tous les acteurs qui ont rendu mon arrivée à Grenoble aussi agréable. Au LJK, je pense notamment à Clément, dont la présence et l'aide est toujours appréciée, à Brigitte, d'une aide précieuse pour les cours d'analyse, et Christophe, qui me guide à travers le système de l'INP et de l'UGA. Merci aussi à Kris, avec qui j'ai commencé à découvrir les environs de Grenoble et qui est toujours de bonne compagnie. Merci à mes colocs, Lola, Yasmine et Eve, qui apportent une joie de vivre et une énergie dont j'ai souvent besoin.

Merci enfin à Philippe et Mohammed pour l'opportunité d'avoir travaillé sur ce sujet et pour m'avoir appris à rédiger des maths. À Nicolas pour son oreille attentive et ses conseils fréquents. À Virginie, pour sa confiance au sujet des cours en GEII. Aux services administratifs qui m'ont rattrapé en urgence pour les démarches de fin de thèse, particulièrement Marie-Aude et Élodie. 
