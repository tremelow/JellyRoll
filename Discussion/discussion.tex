
\clearemptydoublepage
\chapter{Discussion d'extension des résultats}
\label{chap:discussion}

\begin{quote}
    Closure. I keep hearing that word. It's the theater of the absurd. Everybody knows that on television they'll see the end of the story in the last 15 minutes of the thing. It's like a drug. To me, that's the beauty of 'Twin Peaks.' We throw in some curve balls. \textbf{As soon as a show has a sense of closure, it gives you an excuse to forget you've seen the damn thing.}

    \hfill%
    David Lynch, 1990
\end{quote}
\url{https://web.archive.org/web/20200805032143/https://www.latimes.com/archives/la-xpm-1990-02-18-ca-1500-story.html}

\section{Coût de calcul, erreurs d’arrondis, derivative-free, pullback}

Coût de calcul avec les non-linéarités

Difficulté du calcul de la relaxation pour certains schémas

Gain d’ordre avec $z(0) = 0$ (figure avec err sup sur $\varepsilon$)

Donner une clé pour le gain d’ordre : $v_z(0) = \mathcal{O}(\varepsilon)$


\section{Autour de l’équation de télégraphe}

