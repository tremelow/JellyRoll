
\clearemptydoublepage
\chapter{La moyennisation en bref}

Lors de ma troisième année de thèse, j'ai eu l'occasion d'un peu plus me
pencher sur les méthodes de moyennisation, et notamment de rédiger un
mini-article compilant certains résultats du sujet. Une partie est ici,
l'autre sera présentée en Annexe\todo{quelle annexe?}


\section{Présentation d'une méthode}


\subsection{L'équation homologique}

Dérivation de l'équation homologique

Distinction entre averaging standard et stroboscopique


\subsection{Définition d'une décomposition approchée}

Relation de récurrence et résultat sur les bornes

Discussion autour de la méthode pour les résultats (boules imbriquées,
estimations de Cauchy...)



\section{Contexte d'un problème autonome}

\begin{equation}
    \pa_t y = \frac{1}{\eps} G(y) + K(y)
\end{equation}
On décompose
\begin{equation}
    y^\eps(t) = \Omega^\eps _{t/\eps} \circ \Psi^\eps _t 
    \circ \big( \Omega^\eps _0 \big)^{-1}
\end{equation}

\subsection{Un résultat géométrique}

$\Omega$ est un flot, et commute avec $\Psi$. 


\subsection{Cas d'un opérateur linéaire}

Averaging standard : crochets de Lie

Formes normales


\section{Aspect numérique}

Si on ne connaît pas le défaut, il est clair qu'on peut calculer
numériquement 
\begin{empheq}{equation}
    \pa_t v\rk n = F\rk n ( v\rk n ) 
\end{empheq}
et alors 
\begin{empheq}{equation}
    u^\eps(t) = \Phi\rk n _{t/\eps} \big( v\rk n (t) \big)
    + \bigO(\eps^{n+!})
\end{empheq}
Mais en fait si on montre ça, on peut montrer mieux (en supposant un peu de régularité). Section basée principalement sur l'article avec Gilles. 

\subsection{Définition d'un nouveau problème}

Micro-macro ou pullback

Raideur ``retardée''

\subsection{Convergence uniforme}

Résultat de convergence uniforme

Présentation de schémas intégraux et composition de schémas 
