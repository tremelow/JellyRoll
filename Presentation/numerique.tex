\section*{Paradigme numérique}
\addcontentsline{toc}{section}{Paradigme numérique}

En général, on suppose $\eps \ll 1$, et donc le système~\eqref{sec:intro:eq:u} comporte une dynamique \textit{rapide} par rapport au temps d'étude. À cet égard, des méthodes d'\textit{analyse asymptotique} ont été développées, c'est-à-dire des méthodes qui permettent de caractériser le système dans cette limite $\eps$ \enquote{petit}, en général en découplant ces deux dynamiques. Pour les problèmes hautement-oscillants, trois exemples particulièrement célèbres sont les méthodes d'homogénéisation~\cite{goudon.2003.homogenization}, de moyennisation~\cite{perko.1969.higher,sanders.2007.averaging,lochak.1988.multiphase} et de formes normales~\cite{murdock.2006.normal,bambusi.2003.birkhoff}. Pour les problèmes à relaxation rapide, la littérature est moins fournie. Qu'il s'agisse de calculer la variété centrale comme précédemment ou d'un développement de Chapman-Enskog~\textbf{REF}, la phase transitoire n'est pas calculée. 

Plus récemment dans \cite{castella.2016.formal}, les auteurs capturent aussi la phase transitoire, mais la méthode est très difficile à s'approprier et n'est valide que dans la limite $\eps \rightarrow 0$. Dans cette section, on étudie l'application de méthodes numériques \enquote{standards} de l'état de l'art, et on observe le comportement de l'erreur numérique non seulement en fonction du pas de temps $\Dt$, mais aussi en fonction du paramètre $\eps$. 

Dans cette section, on commence par décrire ce qu'on entend par \enquote{méthode numérique} et le contexte dans lequel on va les étudier. Ensuite, on présente trois méthodes d'ordre 2 reconnues dans l'état de l'art~: le splitting de Strang, un schéma IMEX-BDF et une méthode de Runge-Kutta exponentielle. 



\subsection*{Mise en place de la résolution}
\addcontentsline{toc}{subsection}{Mise en place de la résolution}

Pour étudier le comportement des schémas numériques sur les problèmes de la forme~\eqref{sec:intro:eq:u}, on considère l'exemple jouet suivant
\begin{subequations} \label{sec:intro:eq:jouet_v}
    \begin{empheq}[left=\left\lbrace, right=\right.]{align} &
        \pa_t v_1 = v_2 , & &
        v_1(0) = 1 , \vphantom{\frac11}\qquad\qquad
        \\ & 
        \pa_t v_2 = -\frac{1}{\eps}\big( v_1 + v_2 \big) , & &
        v_2(0) = 0 .
    \end{empheq}
\end{subequations}
Cet exemple ressemble à certains problèmes hyperboliques avec relaxation, et sa linéarité le rend simple à étudier. Il prend facilement la forme~\eqref{sec:intro:eq:xz} en posant par exemple $x = v_1$ et $z = v_1 + v_2 $, ce qui donne 
\begin{subequations} \label{sec:intro:eq:jouet_xz}
    \begin{empheq}[left=\left\lbrace, right=\right.]{align} &
        \pa_t x = -x + z , & &
        x(0) = 1 , \vphantom{\frac11}\qquad\qquad
        \\ & 
        \pa_t z = -\frac{1}{\eps}z - x + z , & &
        z(0) = 1 .
    \end{empheq}
\end{subequations}
Ce problème est linéaire et se diagonalise sans problème pour $\eps < 1/4$, ce qui génère
\begin{equation*}
    \newcommand{\sqEps}{\sqrt{1 - 4\eps}}
    \tilde{u} = \underbrace{\begin{pmatrix}
        -1 & 1 - r_\eps \\
        \eps & 1 - \eps - \eps r_\eps 
    \end{pmatrix}}_{\displaystyle P} \begin{pmatrix} x \\ z \end{pmatrix},
    \qquad\text{tel que}\qquad
    \pa_t \tilde{u} = \begin{pmatrix}
        -r_\eps & 0 \\ 0 & -\frac{1}{\eps} + r_\eps
    \end{pmatrix} \tilde{u}
\end{equation*}
avec $r_\eps = \frac{1}{2\eps} \big( 1 - \sqrt{1 - 4\eps} \big)$. On obtient directement une expression explicite pour $u = \among{x}{z}$, qui est 
\begin{equation*}
    u(t) = P^{-1} \begin{pmatrix}
        e^{-t r_\eps} & 0 \\ 0 & e^{-t/\eps} e^{t r_\eps}
    \end{pmatrix} P u(0) 
\end{equation*}
où $P^{-1} = \frac{1}{\sqrt{1 - 4\eps}} \begin{pmatrix}
    -1 + \eps + \eps r_\eps  &  1 - r_\eps \\
                \eps         &      1
\end{pmatrix}$.
\begin{FRremark*}
    On voit bien dans la définition de $r_\eps$ que le problème change de nature entre $\eps \leq 1/4$ et $\eps > 1/4$. En effet, dans le premier cas le système est purement dissipatif, alors que dans le second, des oscillations apparaissent. Cette singularité apparaît également dans la matrice de changement de variable, dont le déterminant vaut $-\sqrt{1 - 4\eps}$.
\end{FRremark*}


\paragraph{Introduction aux résolutions numériques\\}
Dans les faits, on ne saura pas résoudre tous les systèmes de la forme~\eqref{sec:intro:eq:u} de manière exacte. Donc on va appliquer des méthodes d'\textit{approximation numérique} pour calculer une solution approchée. À partir l'exemple~\eqref{sec:intro:eq:jouet_xz}, on va étudier la précision de ces méthodes. Voici la manière dont on procède~:

Discrétisation de $t$ de manière uniforme

Def : erreur d’une méthode numérique 

Rq : on pourrait considérer une interpolation des données et définir une erreur dans le monde continu mais c’est compliqué et déjà si l'erreur ponctuelle est bien on est content

On suppose qu’on sait calculer $\exp(-t A)$ pour Strang ou expRK, ou qu’on sait facilement inverser $\id + \Dt \, A$ pour IMEX-BDF.

Def : ce qu’on entend par un schéma numérique, $$u_{n+s} = u_{n+s-1} + \Dt\, \Phi^{\eps}_{\Dt} (u_{n+s-1}, \ldots, u_n)$$

Footnote : en vérité, quitte à remplacer $u_n$ par $U_n = (u_{n+s-1}, \ldots, u_n)$, on peut se ramener à des méthodes à une seule étape


\subsection*{Méthodes numériques}
\addcontentsline{toc}{subsection}{Méthodes numériques}

On présente les résultats associés à trois méthodes d'ordre 2, qui traitent la partie raide différemment de la partie non-raide. Les méthodes sont bien définies  dans la limite $\eps \rightarrow 0$.

Attention : on observe le comportement de l’erreur et sa relation avec $\Dt$ mais aussi avec $\eps$ !

Disclaimer : je n’ai pas étudié les schémas en eux-mêmes, je les ai juste compilés et ai étudié leur comportement

Rq : Je n’ai pas étudié de méthodes complètement implicites parce qu'elles sont très coûteuses, pas très adapté pour l'extension aux EDP... Les méthodes purement explicites demandent $\Dt < \eps$, ce qui est beaucoup trop coûteux. On cherche justement à se débarasser de cette contrainte. 

\paragraph{Splitting de Strang\\}

Une approche courante est de séparer le problème~\eqref{sec:intro:eq:u} en deux parties, une raide et une non-raide. La manière naturelle de procéder fournit
%
\begin{empheq}[left=\left\lbrace, right=\right.]{align*} &
    \pa_t u^{(1)} = -\frac{1}{\eps}A u^{(1)} ,
    \\ &
    \pa_t u^{(2)} = f(u^{(2)}) . \vphantom{\frac11}
\end{empheq}
%
On note $\varphi_t$, $\varphi^{(1)}_t$ et $\varphi^{(2)}_t$ les $t$-flots associés aux problèmes en $u$, $u^{(1)}$ et $u^{(2)}$ espectivement. On remarque qu'il est simple de calculer $\varphi^{(1)}$ de manière exacte, et simple de calculer $\varphi^{(2)}$ de manière numérique. Cependant, ces deux dynamiques sont mélangées dans $\varphi$, ce qui rend le flot du problème d'origine difficile à calculer. Ainsi, on est en droit de se poser la question~: Est-il possible d'obtenir $\varphi$ à partir de $\varphi^{(1)}$ et de $\varphi^{(2)}$~? 

La réponse est non en général, mais on peut \textit{approcher} $\varphi$ à partir des autres avec des compositions successives. C'est cette approche qu'on appelle \textit{splitting}. Le plus couramment utilisé est le splitting de Strang, qui s'écrit 
\begin{equation*}
    \varphi_t = \varphi^{(1)}_{t/2} \circ \varphi^{(2)}_{t} \circ \varphi^{(1)}_{t/2} + \bigO(t^3) .
\end{equation*}
Pour la plupart des équations, l'ordre des opérations n'a pas d'importance, mais lorsque le système présente une partie de relaxation raide comme ici, il a été remarqué dans~\cite{sportisse.2000.analysis,descombes.2004.operator} qu'il vaut mieux \enquote{terminer} par la relaxation. 
Ce schéma peut être obtenu par symétrie à partir du splitting de Lie $\Phi_t = \varphi^{(2)}_{t} \circ \varphi^{(1)}_{t}$, d'ordre 1. 
% Attention, $\Phi_t$ n'est pas un $t$-flot au même sens que $\varphi_t$, c'est la solution particulière d'une EDO dans l'espace des morphismes. Dans le cas où $f$ est linéaire, on dispose de la formule de Baker-Campbell-Hausdorff (BCH),\todo{REF BCH}
% \begin{equation*}
%     \log \big( \Phi_t \big)
%     = t\left(\frac{-1}{\eps}A + f\right) 
%     - \frac{t^2}{2\eps} \left[A, f\right]
%     + \frac{t^3}{12\eps^2}\big( [A,[A,f]] + \eps [f,[A,f]] \big)
%     + \ldots
% \end{equation*}
% avec $[f,g] = fg - gf$ le commutateur de champs linéaires. Cette formule n'est valide que formellement, c'est-à-dire qu'elle ne converge pas forcément et que l'important est le terme général qui la génère. On peut la comparer à un développement de Taylor qu'on a poussé à un ordre \enquote{infini}. L'objet obtenu est une série entière en $t$, mais celle-ci peut avoir un rayon de convergence nul. Néanmoins, les termes de la série ont du sens et on peut en prendre les premiers termes pour construire des approximations, bien qu'il faille être prudent avec la régularité de la fonction pour obtenir des résultats rigoureux.
Le splitting est exact si et seulement si les champs $A$ et $f$ commutent, c'est-à-dire 
\begin{equation*}
    Af - \pa_u f \cdot A = [A,f] = 0 .
\end{equation*}
Dans ce cas, le splitting de Lie génère un flot qui coïncide avec $\varphi$. Évidemment, ce n'est pas le cas en général. En particulier dans le cas test~\eqref{sec:intro:eq:jouet_xz}, on a $[A,f] = \begin{pmatrix} 0 & -1 \\ -1 & 0 \end{pmatrix}$. 
% \begin{itemize}
%     \item Le morphisme $\Phi_t$ est un flot;
%     \item $\Phi_t$ et $\varphi_t$ coincïdent en tout $t$;
%     \item Les champs $A$ et $f$ commutent.
% \end{itemize}
% Le morphisme $\Phi_t$ est un flot si et seulement si cette expression est linéaire en $t$, or ce n'est pas le cas en général. Le seul moyen que ce soit le cas est que $A$ et $f$ commutent. Ce résultat est aussi valide dans le cas non-linéaire, avec le commutateur
% \begin{equation*}
%     [f,g] = \pa_u f \cdot g - \pa_u g \cdot f 
% \end{equation*}
% qui définit une algèbre de Lie sur les champs de vecteurs. Ceci correspond à une manière plus géométrique de considérer les champs de vecteurs qui revient à les associer à l'opérateur de transport $\mathcal{D}_f(g) = \pa_u g \cdot f$. Ainsi, $[f,g] = \mathcal{D}_g(f) - \mathcal{D}_f(g)$. 

% On définit l'erreur de splitting 
% \begin{equation*}
%     \err_{\mathrm{Lie}} 
%     = \log \big( \Phi_t \big) - t\left(\frac{-1}{\eps}A + f\right) 
% \end{equation*}
% qui permet d'obtenir l'erreur de troncature du schéma par un passage à l'exponentielle. Il apparaît à partir de la formule de BCH que cette erreur est de taille $t^2/\eps$, pas indépendante de $\eps$. Néanmoins, on a déjà vu avec Euler implicite que l'accumulation d'erreur pouvait devenir indépendante de $\eps$, grâce aux propriétés de décroissance de $z$. C'est le cas ici, et on peut borner l'erreur \textit{indépendamment} de $\eps$. 

% On cherche ensuite à améliorer la convergence du schéma: il serait agréable de pouvoir profiter d'une convergence à un ordre plus élevé. La méthode généralement considérée est le splitting de Strang~:
% \begin{equation*}
%     \varphi_t \approx \varphi^{(1)}_{t/2} \circ \varphi^{(2)}_{t} \circ \varphi^{(1)}_{t/2} .
% \end{equation*}
% Cette méthode a l'avantage d'être d'ordre 2, et de présenter des propriétés géométriques sympathiques de par sa symétrie (elle génère d'ailleurs la méthode de Störmer-Verlett, voir Annexe~\textbf{REF}\todo{Storm-Verl}). Néanmoins, il n'est pas clair qu'elle présente un bon comportement lorsque la raideur augmente, i.e. lorsque $\eps$ diminue. En effet, un calcul de l'erreur de troncature donne 
% \begin{equation*}
%     \varphi_t - \varphi^{(1)}_{t/2} \circ \varphi^{(2)}_{t} \circ \varphi^{(1)}_{t/2} 
%     = \frac{t^3}{12\eps^2} \big( [A,[A,f]] + \eps [[A,f],f] \big)
%     + \bigO(t^4) .
% \end{equation*}
% Ainsi, même si un $1/\eps$ est compensé par la décroissance rapide de $z$, l'erreur ne sera pas indépendante de $\eps$. On peut vérifier ce résultat de manière numérique.

\todo[inline]{Figure de convergence en fonction de $\Dt$ et $\eps$, avec Lie en haut et Strang en bas. Séparer $x$ et $z$?}

Dans cette figure, on observe que le comportement de la solution est le bon attendu pour $\Dt \ll \eps$. Néanmoins, lorsqu'on trace l'erreur en fonction de $\eps$, on voit qu'à $\Dt$ fixé, il y a toujours un seuil à partir duquel une réduction de $\eps$ entraîne une augmentation de l'erreur. Cette augmentation, de tendance prédite en $1/\eps$, entraîne une \textit{réduction d'ordre}, c'est-à-dire qu'il n'y a pas de constante d'erreur $C$ telle que 
\begin{equation*}
    \sup_\eps \err \leq C \Dt^2 .
\end{equation*}

\bigskip\bigskip\bigskip
\todo[inline]{Annexe B~: Stormer-Verlett est un cas particulier de Strang mélangé à Euler explicite. En effet avec $\dot q = v$ et $\dot v = F(q)$,}
\vspace*{-2\bigskipamount}
\begin{align*}
&   v_{n + 1/2} = v_n + \frac{\Dt}{2}F(q_n) \\
&   q_{n+1} = q_n + \Dt v_{n+1/2} \\
&   v_{n+1} = v_{n+1/2} + \frac{\Dt}{2}F(q_{n+1}) .
\end{align*}
\vspace*{-\medskipamount}
\todo[inline]{Ce qui revient à séparer le système en $\pa_t (q,v) = (0, F(q))$ et $\pa_t (q,v) = (v, 0)$.}

Présentation rapide du splitting de Lie et de Strang, avec une résolution exacte des flots

Fig : convergence de Lie ($\Dt$ à gauche, $\eps$ à droite)

Fig : convergence de Strang ($\Dt$ à gauche, $\eps$ à droite)

Rq : réduction d’ordre notée dans un article de Sportisse en 2000 \url{https://www.sciencedirect.com/science/article/pii/S0021999100964957}


\paragraph{Méthode IMEX-BDF}

Justification de l’utilisation de cette méthode avec le côté “UA” et les IMEX-LM en cinétique

Formule de IMEX-BDF Euler avec justif

Fig : convergence de la méthode ordre 1

Fig : convergence de la méthode ordre 2


\paragraph{Méthodes exponentielles Runge-Kutta}

Formulation intégrale avec semi-groupe

Justification de l’utilisation de cette méthode avec la norme “relative”

Formule de expRK Euler

Rq : aussi considéré dans un article de Sportisse (\url{https://ir.cwi.nl/pub/4597}) pour compenser les failles d’erreur du splitting

Fig : convergence méthode ordre 1

Fig : convergence méthode ordre 2


\subsection*{Analyse des résultats}
\addcontentsline{toc}{subsection}{Analyse des résultats}

Toutes les méthodes se comportent de la même manière… Il y a réduction d’ordre

Def : convergence AP

Def : convergence UA

\textbf{Rq~:} Souvent, la littérature parle de convergence UA lorsqu'il s'agit d'une convergence UA \enquote{faible}, i.e. avec une donnée bien préparée $z(0) = \eps h^\eps(x(0)) + \bigO(\eps^n)$. 
