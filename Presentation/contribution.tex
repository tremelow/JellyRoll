\section*{Contribution personnelle}
\addcontentsline{toc}{section}{Contribution personnelle}


Suite aux résultats de précision uniforme (i.e. indépendante de $\eps$) obtenus pour les problèmes hautement oscillants~\cite{chartier.2015.uniformly,chartier.2020.new,crouseilles.2017.nonlinear}, on étudie dans cette section quelques méthodes numériques à la pointe de l'état de l'art qui concernent 

Résultat de convergence uniforme avec IMEX-BDF ou expRK

Explication de la méthode à détailler (sans prendre le crédit)~:
\begin{itemize}
    \item Lien avec moyennisation
    \item Formes normales, splitting quasi-exact
    \item Conservation exacte du défaut
\end{itemize}

Ouverture au cinétique

Annonce plan