\section*{Contribution personnelle}
\addcontentsline{toc}{section}{Contribution personnelle}


Suite aux résultats de précision uniforme obtenus pour les problèmes hautement oscillants~\cite{chartier.2015.uniformly,crouseilles.2017.nonlinear,chartier.2020.new}, ce travail de thèse a cherché à développer des méthodes à précision uniforme dans le cadre des problèmes à relaxation rapide de type~\eqref{sec:intro:eq:u}. 
Il existe deux grandes stratégies pour obtenir une convergence uniforme. La première est le développement double échelle, où on écrit la solution $t \mapsto u(t)$ comme une évaluation particulière d'une fonction à deux variable $(t,\theta) \mapsto U(t,\theta)$ en posant
\begin{equation*}
    u(t) = U(t,\theta)|_{\theta = t/\eps} .
\end{equation*}
L'apparition de cette seconde variable permet de choisir une donnée initiale $U(0,\theta)$ qui réduit la raideur dans la direction $t$. La seconde stratégie est la décomposition micro-macro. 
%
%
% Les premiers résultats de précision uniforme proviennent d'un développement double-échelle, c'est à dire qu'on écrit la solution $t \mapsto u(t)$ comme une évaluation particulière d'une fonction à deux variable $(t,\tau) \mapsto U(t,\tau)$ en posant
% \begin{equation*}
%     u(t) = U(t,\tau)|_{\tau = t/\eps} .
% \end{equation*}
% L'apparition de cette seconde variable permet de choisir une donnée initiale $U(0,\tau)$ qui réduit la raideur dans la direction $t$. Les premiers mois de cette thèse ont été passés à adapter ces résultats dans le cadre des problèmes à relaxation rapide. Malheureusement, malgré une méthode théoriquement convergente, l'implémentation efficace du développement double-échelle n'a pas été possible.\footnote{La difficulté peut se résumer à cela~: Dans le cadre de problèmes hautement oscillants, la donnée double-échelle $U(t,\tau)$ est construite périodique par rapport à $\tau$. Dans le cadre d'une relaxation rapide, une telle construction n'est pas possible et on doit considérer $\tau$ dans $\setR_+$, ou au moins dans $[0,T/\eps]$. Résoudre une équation de transport sur un domaine si grand pose des difficultés majeures.} Néanmoins, ces résultats seront présentés en Annexe~\ref{chap:two-scale}, de sorte à fournir des clés dans le cas où une personne souhaiterait élaborer des méthodes double-échelle pour ce genre de problèmes.
%
%
L'idée est similaire à celle du double-échelle; il s'agit de séparer la dynamique rapide oscillante (en $e^{it/\eps}$) et la dynamique de \textit{dérive} (en $t$). Ainsi on écrit 
\begin{equation*}
    u(t) = \Omega^\eps_{t/\eps} \circ \Gamma^\eps_t \circ \big( \Omega^\eps_0 \big)^{-1} (u_0)
\end{equation*}
où $\theta \mapsto \Omega^\eps_\theta$ est périodique et $\Gamma^\eps_t$ est le $t$-flot d'un champ de vecteurs non-raide $F^\eps$. C'est cette seconde approche que nous avons privilégiée, puisqu'elle permet de ne pas avoir à considérer de seconde variable $\theta$ lors de la résolution numérique et d'utiliser des schémas standards. Néanmoins, on fournit en Annexe~\ref{chap:two-scale} des pistes pour adapter un développement double-échelle au cadre dissipatif. 

L'idée de la méthode micro-macro est de chercher des approximations de $\Omega^\eps$ et de $F^\eps$ à un certain ordre $\eps^{n+1}$ près, dans la veine des méthodes de moyennisation~\cite{perko.1969.higher,lochak.1988.multiphase,sanders.2007.averaging,chartier.2010.higher,chartier.2012.formal,castella.2015.stroboscopic}. La nouveauté ici est de s'intéresser en outre au reste de ce développement asymptotique, ainsi on obtient une décomposition exacte 
\begin{equation*}
    u(t) = \Omega\rk{n}_{t/\eps} \big( v(t) \big) + w(t)
\end{equation*}
avec $w(t) = \bigO(\eps^{n+1})$ et $\pa_t v = F\rk n(v)$. Les applications $\Omega\rk n$ et $F\rk n$ sont des approximations de $\Omega^\eps$ et $F^\eps$ respectivement. Dans~\cite{chartier.2020.new}, le problème sur $(v,w)$ est moins raide et peut être résolu avec un schéma numérique d'ordre uniforme~$n$. 


L'utilisation de ces méthodes de moyennisation a permis, au cours de la troisième année de thèse, de dériver les résultats préexistants en moyennisation avec certaines preuves originales, notamment sur le caractère géométrique de la moyennisation dite \textit{stroboscopique}. Précédemment, les démonstrations demandaient de construire le morphisme $\Omega^\eps$ et le champ de vecteurs moyen $F^\eps$, pour ensuite raisonner par récurrence ou avec des arbres. Ces nouvelles preuves ne font appel qu'à l'équation homologique (i.e. l'équation algébrique vérifiée par $\Omega^\eps$ et $F^\eps$), ce qui permet une meilleure compréhension des structures impliquées. En outre, on met en évidence certains liens forts entre moyennisation et formes normales, qui sont déjà connus (voir~\cite{sanders.2007.averaging}) mais peu référencés. Cette synthèse fait l'objet du Chapitre~\ref{chap:avg}. 


L'essentiel de ce travail de thèse a consisté à construire des méthodes micro-macro adaptées aux problèmes à relaxation rapide de type~\eqref{sec:intro:eq:u}. Formellement, au lieu de voir le changement de variable $\Omega^\eps_\theta$ comme une série de Fourier en $\theta \in \setR/\setZ$, il est vu ici comme une série exponentielle $\Omega^\eps_\tau = \sum_{k \geq 0} \omega_k e^{-k\tau}$ avec $\tau \in \setR_+$. La nouvelle difficulté est alors de calculer l'équivalent formel de la \enquote{moyenne} de cette série exponentielle. Pour cette raison, on considère $\Omega^\eps_{i\tau}$ et on adapte les résultats du cas périodique. En effectuant des développements à l'ordre $n$, on obtient ainsi un problème micro-macro en $(v,w)$ qu'on peut résoudre avec une précision uniforme d'ordre $n+1$ (le caractère de relaxation permet un gain d'ordre par rapport aux problèmes hautement oscillants) en utilisant des schémas expRK. On peut aussi obtenir une précision uniforme d'ordre $n$ avec des schémas IMEX-BDF. Il est intéressant de noter que ce résultat est étendu partiellement à des EDP bien connues~: les problèmes hyperboliques relaxés
\begin{equation*}
    \left\{ \begin{array}{l} \displaystyle
    \pa_t v_1 + \pa_x v_2 = 0 , \\ \displaystyle
    \pa_t v_2 + \pa_x v_1 = \frac{1}{\eps} \big( g(v_1) - v_2 \big) ,
    \end{array} \right .
\end{equation*}
et l'équation de télégraphe (i.e. BGK à vitesses discrètes)
\begin{equation*}
    \left\{ \begin{array}{l} \displaystyle
    \pa_t \rho + \pa_x j = 0, \\ \displaystyle
    \pa_t j + \frac{1}{\eps}\pa_x \rho = -\frac{1}{\eps} j .
\end{array} \right .
\end{equation*}
De nombreuses méthodes AP ou UA à l'équilibre existent pour ces problèmes --on peut citer~\cite{jin.1999.efficient,lemou.2008.new,dimarco.2011.exponential,dimarco.2017.implicit,boscarino.2017.unified,albi.2020.implicit}-- mais le développement de méthodes UA est encore un sujet de recherche actif. On peut voir ce travail de thèse comme une étape préliminaire au développement de méthodes UA pour cette catégorie de problèmes. Ces résultats ont fait l'objet d'une publication, 
\begin{center}\begin{minipage}{.75\textwidth}
    \noindent
    \fullcite{chartier.2021.uniformly},
\end{minipage}\end{center}
qui est présentée en Chapitre~\ref{chap:dissip-mima}. 

En Chapitre~\ref{chap:discussion}, on discute d'extensions directes des résultats de notre article, qui servent à rendre le résultat plus robuste, et on fournit quelques pistes pour traiter l'équation de télégraphe de manière complète. 