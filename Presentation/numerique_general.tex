\section*{Application de méthodes numériques générales}
\addcontentsline{toc}{section}{Application de méthodes numériques générales}

Cette section observe le comportement de méthodes numériques usuelles
lorsqu'on les applique au système.

\subsection*{Application directe de schémas standards}
\addcontentsline{toc}{subsection}{Application directe de schémas standards}


On cherche à résoudre le problème~\eqref{sec:intro:eq:u} numériquement.
En termes simples, cela veut dire trouver une méthode générale pour
obtenir la solution du problème. On va montrer que les méthodes
\enquote{standards} sont inefficaces pour résoudre même le problème
simplifié 
\begin{equation} \label{sec:intro:eq:pb_zlin}
    \pa_t z = -\frac{1}{\eps}z
\end{equation}
avec une donnée initiale $z(0) \in \setR$ quelconque. Par
\enquote{méthodes standards}, on entend des méthodes qui peuvent être
trouvées dans le livre de référence \textbf{REF}\todo{Hairer Wanner 1}.
Il s'agit de méthodes de résolution pour des équations différentielles
ordinaires quelconques, qui ne prennent pas en compte la structure
spécifique du probléme qui nous intéresse. 


Il est difficile de traiter le continu d'un point de vue numérique, donc
on commence d'abord par \textit{discrétiser} l'intervalle de temps en un
nombre arbitraire $N \in \setN^*$ d'intervalles.
%
\todo[inline]{Dessin avec la définition des $t_n$ (voir commentaire)
    % |-----|-----|---//---|-----|
    % 0     Dt   2Dt      T-Dt   T
}
%
\noindent%
Pour le moment, on choisit de se restreindre à une discrétisation
uniforme, c'est-à-dire que l'intervalle de temps $[0,T]$ est divisé en
$N$ intervalles de taille égale. De manière équivalente, on définit les
points de séparation $(t_n)_{0 \leq n \leq N}$ avec 
\begin{equation*}
    t_n = \frac{n}{N} T ,
\end{equation*}
ou encore 
\begin{equation*}
    t_{n+1} = t_n + \Dt
    \qquad\text{avec}\qquad
    t_0 = 0 \quad\text{et}\quad \Dt = \frac{T}{N} .
\end{equation*}

L'idée est maintenant d'obtenir une approximation de~$u_n \approx
u^\eps(t_n)$ en chaque point. La méthode la plus directe est la méthode
d'Euler~: On a accès à la valeur initiale $u_0$, et on peut ensuite
calculer le prochain point par projection 
%
\todo[inline]{\textbf{Graphique :} Tracer la droite qui part de $z_n$
avec la pente $-z_n/\eps$ pour placer le point $z_{n+1}$.}
%
\noindent%
On obtient ainsi la \textit{suite d'Euler} du
problème~\eqref{sec:intro:eq:pb_zlin}~:
\begin{equation*}
    z_{n+1} = \left( 1 - \frac{\Dt}{\eps} \right) z_n , 
    \qquad
    z_0 = z(0) .
\end{equation*}
D'autres méthodes de projection existent, notamment les méthodes de
Runge-Kutta qui impliquent des points intermédiaires, ou les méthodes
multi-points qui utilisent les valeurs en $n, n+1, \ldots, n+s$ pour
calculer le point en $n+s+1$. Le raisonnement qui suit s'applique (à une
constante près) à toutes ces méthodes de calcul tant qu'elles sont
\textit{explicites}. Le schéma numérique est dit \textit{stable} si la
norme de la solution calculée ne croît pas,\footnote{%
Cette définition atteint vite sa limite lorsqu'on considère des
problèmes à solution croissante, par exemple $\pa_t x = x$.
Heureusement, ce n'est pas le cas ici.} i.e. si 
\begin{equation*}
    \left|\, 1 - \frac{\Dt}{\eps} \,\right| \leq 1 .
\end{equation*}
Cette condition est généralement interprété comme une restriction sur le
pas de temps $\Dt$. Néanmoins, nous sommes aussi intéressés par le
comportement du schéma par rapport à la variable de raideur $\eps$. En
considérant $\Dt$ fixé, il est clair qu'il faut prendre $\Dt$ plus petit
que $\eps$ pour avoir une solution décroissante. Le coût de calcul dans
la limite $\eps \rightarrow 0$ évolue ainsi au moins en $\bigO(1/\eps)$.

Un moyen généralement invoqué pour stabiliser les problèmes raides comme
celui-ci est l'utilisation de schémas \textit{implicites}. Au lieu de
faire une projection directe, on cherche le point qui, en temps
rétrograde, aurait donné le point précédent.
%
\todo[inline]{Schéma avec différentes lignes de niveau et les tangentes
qui partent de différents $u_{n+1}$ possibles}
%
\noindent%
Cette méthode de projection rétrograde \enquote{directe} est appelée
méthode d'Euleur implicite, et le schéma associé dans le cas du
problème~\eqref{sec:intro:eq:pb_zlin} est 
\begin{equation*}
    z_{n+1} = z_n - \frac{\Dt}{\eps} z_{n+1} ,
\end{equation*}
ou encore 
\begin{equation*}
    z_{n+1} = \frac{\eps}{\Dt + \eps} z_n .
\end{equation*}
En comparant les figures~\textbf{REF}\todo{Labels}, il est clair que
cette méthode est en général bien plus coûteuse que les méthodes
explicites de projection directe. Néanmoins, elle présente quelques
propriétés sympathiques que nous devons présenter. Ce schéma est
clairement stable pour tout $\Dt$ et tout $\eps$, puisque le facteur
multiplicatif est toujours inférieur à $1$. Cependant, cela n'indique
rien sur l'erreur associée au schéma. Calculons-la rapidement.

On pose $\tau_n$ l'erreur dite de troncature du schéma entre $t_n$ et
$t_{n+1}$,
\begin{equation*}
    \tau_n = z(t_{n+1}) - z(t_n) - \Dt \pa_t z(t_{n+1}) .
\end{equation*}
% Par théoréme fondamental de l'analyse, $\tau_n = \int_0^{\Dt} \big(
% \pa_t z(t_n + s) - \pa_t z(t_{n+1}) \big) \dd s$, d'où, avec une
% application supplémentaire de la formule,
% \begin{equation*}
%     \tau_n = \int_0^{\Dt} \int_s^{\Dt} 
%         \pa_t^{\:2} z(t_n + \mu) \dd \mu \dd s
% \end{equation*}
% On remarque qu'il s'agit d'une intégrale sur le triangle~:
% \todo[inline]{Figure du triangle (voir commentaire)
% % (0,h)  ----------- (h,h)
% %        |        /
% %  mu    |      /
% %        |    /
% %        |  /
% % (0,0)  |/
% %            s
% }
% On peut donc inverser les intégrales et obtenir
% \begin{equation*}
%     \tau_n = \int_0^{\Dt} \int_0^{\mu} 
%         \pa_t^{\:2} z(t_n + \mu) \dd s \dd \mu .
% \end{equation*}
% d'où la borne par inégalité triangulaire 
En général on borne cette erreur par une application successive
d'intégration par parties, inégalité triangulaire et enfin inégalité de
Hölder, pour trouver
\begin{equation} \label{sec:intro:eq:euler_err1}
    \left| \tau_n \right| 
    \leq \int_0^{\Dt} s \left| \pa_t^{\:2} z(t_n + s) \right| \dd s
    \leq \Dt \int_{t_n}^{t_{n+1}} \left| \pa_t^{\:2} z(t) \right| \dd t.
\end{equation}
En revanche cette approche n'exploite pas le caractère décroissant de $t
\mapsto z(t)$, et d'ailleurs cette dernière borne serait aussi valide
pour le schéma d'Euler explicite. Cette caractéristique peut être
exploitée en s'intéressant directement à la définition de~$\tau_n$. En
effet dans notre cas particulier, on a $z(t) = e^{-t/\eps} z_0$ donc 
\begin{equation*}
    \tau_n = \left(
        e^{-\Dt/\eps} - 1 + \frac{\Dt}{\eps}e^{-\Dt/\eps}
        \right) e^{-n \Dt/\eps} z_0 ,
\end{equation*}
ce qui peut se borner très grossièrement,
\begin{equation} \label{sec:intro:eq:euler_err2}
    |\tau_n| \leq e^{-n \Dt/\eps} |z_0|
\end{equation}
On remarque maintenant que l'erreur globale vérifie 
\begin{equation*}
    \left(1 + \frac{\Dt}{\eps}\right) \left( z_{n+1} - z(t_{n+1})\right)
    = z_n - z(t_n) - \tau_n 
\end{equation*}
d'où de proche en proche,
\begin{equation*}
    \left| z_{n+1} - z(t_{n+1}) \right| 
    \leq \sum_{k = 0}^n 
        \left(1 + \frac{\Dt}{\eps}\right)^{-k-1} |\tau_{n-k}| .
\end{equation*}
On applique brutalement une inégalité de Hölder sur la somme pour
obtenir finalement 
\begin{equation*}
    \left| z_{n+1} - z(t_{n+1}) \right| 
    \leq \left(1 + \frac{\Dt}{\eps}\right)^{-1} 
        \sum_{k = 0}^n |\tau_{k}| .
\end{equation*}
En exploitant~\eqref{sec:intro:eq:euler_err1}
et~\eqref{sec:intro:eq:euler_err2}, on obtient enfin 
\begin{equation*}
    \left| z_{n+1} - z(t_{n+1}) \right| 
    \leq \frac{\eps}{\eps + \Dt} \min\left( 
        \Dt \| \pa_t^{\:2} z \|_{L^1} ,
        \frac{ |z_0| }{ 1 - e^{-\Dt/\eps} } \right) .
\end{equation*}
Ici la norme $L^1$ est définie de manière canonique, c'est-à-dire $ \| u
\|_{L^1} \leq \int_0^T | u(t) | \dd t $. En l'occurrence, on connaît
exactement $z(t) = e^{-t/\eps} z_0$, donc on peut calculer $\|
\pa_t^{\:2} z \|_{L^1} = \frac{1}{\eps}(1 - e^{-T/\eps}) z_0$, ce qui
donne finalement pour tout $n$ entre $0$ et $N$, 
\begin{equation*}
    \left| z_n - z(t_n) \right| 
    \leq \frac{\Dt}{\eps + \Dt} |z_0| \min \left(
        1, \frac{\eps/\Dt}{1 - e^{-\Dt/\eps}}
    \right) .
\end{equation*}
Grâce au premier argument du $\min$, on est assuré que l'erreur est de
la forme $C\Dt$ pour tout $\Dt < \Dt_0$ avec $C$ et $\Dt_0$ des
constantes indépendantes de $\eps$. On remarque là une différence par
rapport au comportement du schéma explicite, qui demandait $\Dt \ll
\eps$. 
%
Le deuxième argument du $\min$ nous assure que l'erreur tend vers zéro
quand $\eps$ tend vers zéro. En particulier, on dit le schéma préserve
le comportement asymptotique,\footnote{Ce schéma est en fait $L$-stable,
ce qui est une notion un peu plus forte qui ne nous intéresse pas dans
le cadre de ce manuscrit.} ce qui se traduit en \textit{asymptotic
preserving} (AP) en anglais. Cette notion reviendra dans la prochaine
section. 



\bigskip\bigskip\bigskip
\textbf{Rq~:} On peut obtenir une approximation en tout point $t \in
[0,T]$ par interpolation. Le sujet de cette thèse n'est pas l'étude de
schémas numériques ou celle de méthodes d'interpolation, donc nous
éviterons de fournir des détails superflus sur ces sujets. On évaluera
la qualité d'une approximation en fonction de la valeur aux points. 

Illustration rapide sur méthode d'Euler

Démonstration d'erreur en $\Dt/\eps$

Distinction entre convergence et stabilité



\subsection*{Méthodes générales pour systèmes raides}
\addcontentsline{toc}{subsection}{Méthodes générales pour systèmes raides}

Bouquin de Hairer, Lubich, Wanner

Présentation rapide de méthodes stiff (implicites)

Discussion du coût du pas de temps adaptatif

Splitting + simulations

Foreshadowing~: ordre des opérations pour AP et splitting exact pour
formes normales