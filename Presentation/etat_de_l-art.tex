\section*{Exposition de l'état de l'art}
\addcontentsline{toc}{section}{Exposition de l'état de l'art}

Cette section observe le comportement de méthodes numériques usuelles lorsqu'on les applique au système.

\subsection*{Méthodes avec traitement spécifique de la partie raide}
\addcontentsline{toc}{subsection}{Méthodes avec traitement spécifique de la partie raide}

IMEX-BDF, convergence théorique

expRK, convergence théorique

Rq : Lawson et IMEX-RK, pas particulièrement étudiées

Test des méthodes proches et loin de l'équilibre

\subsection*{Notions de convergence}
\addcontentsline{toc}{subsection}{Notions de convergence}

Def : AP (faire un diagramme)

Def : UA restreint

Def : UA tout court (trouver un diagramme)

Introduction de la norme des schémas exponentiels
(notion de norme relative)

\subsection*{Contribution personnelle}
\addcontentsline{toc}{subsection}{Contribution personnelle}

Résultat de convergence uniforme avec IMEX-BDF ou expRK

Explication de la méthode à détailler (sans prendre le crédit)~:
\begin{itemize}
    \item Lien avec moyennisation
    \item Formes normales, splitting quasi-exact
    \item Conservation exacte du défaut
\end{itemize}

Ouverture au cinétique

Annonce plan